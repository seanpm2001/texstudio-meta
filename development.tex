\documentclass[10pt,a4paper,landscape]{report}
\usepackage[utf8]{inputenc}
\usepackage{amsmath}
\usepackage{amsfonts}
\usepackage{amssymb}
\usepackage{geometry}

\begin{document}
\chapter{Rules}
This chapter describes some development rules.

\section{Basic C++}
The identifier should be named in these ways:\\
\begin{tabular}{ll}
ClassSomething & First character and every word in uppercase for a \emph{class}\\
identifierSomething & camelCase for all local variables and \emph{members} (notice: no m\_ prefix for texmaker compatibility)\\
headersomething.h & All header in lowercase \\
headersomething.cpp & sources, too \\
HEADERSOMETHING\_H & include guards in uppercase \\
SOME\_CONSTVALUE & const/enum values in uppercase ?????? (with prefix??? without???)
\end{tabular}

Order the include directives from high level to low level, e.g.:\\
\begin{verbatim}
#include "texmaker.h"

#include <qt>
\end{verbatim}
instead of the other way around

Insert only things in texmaker.cpp or smallUsefulFunction.h if it is really necessary,
better create a new .cpp/.h if you can't find one to put it in

\section {Compatibility}
Don't use QT 4.4 or 4.5, only functions of Qt4.3.\\
If they are really necessary you could wrap them in \#if, like: \\
\verb+#if QT_VERSION >= 0x040500+ für Qt 4.5.0\\
\verb+#if QT_VERSION >= 0x040400+ für Qt 4.4.0\\

But this prevents linking with older qts, so it is better  to call hasAtLeastQt(4,5) of smallAndUseful, to check and then call the methods dynamically.\\
Use QMetaObject::invokeMethod for Slots/Signals and QObject::setProperty for properties

Some can be easily replaced, these include:

\begin{center}
\begin{tabular}{lll}
example & use instead & remark\\
\hline \\
QFormLayout & QGridLayout\\
Q(String)List.append(Q(String)List)  &  $<<$ & This doesn't apply to .append(T) \\
Q(String)List.length  &  QList.count() \\
Q(String)List.removeOne   &  \verb!if (i=QList.indexOf()>0) QList.removeAt(i)! \\
QRegExp.cap const  & QRegExp.cap & cap has first been made const in 4.5 \\
\end{tabular}
\end{center}

Those don't exists in old qt, but should be used for the newer version, so they must be included in \#if.
(min version is the minimum version where it can be used)
\begin{tabular}{llll}
example & min version & possible replacement\\
QPalette::ToolTipBase & 4.4 & QPalette::AlternateBase\\
QPalette::ToolTipText & 4.4 & QPalette::Text
\end{tabular}

\chapter{TODO}

\subsection{necessary before 2.0 }

\subsubsection{Bugs}
\begin{itemize}
\item \verb+\irgendwas+ wird in der unittest ausgabe rückwärts las \verb+sawdnegri\+ geschrieben?? (Still wichtig?)
	\item codesnippt mit placeholder einfügen und rückgängig machen plaziert cursor an falscher column position => schreiben zerstört rückgängig
	\item check support for Chinese (bug 2854917, forum)
\end{itemize}

\subsubsection{others}
\begin{itemize}
\item look at texmaker 2.0 (especially quickbuild)
\item larger (=sharp) icons for the taskswitcher (???)
\item Tests
\end{itemize}


\subsection{later}

\subsubsection{Bugs}
\begin{itemize}
\item  preview: requires saved tex file?  
\item  writing \verb+\abstract"+ creates \verb+\abstractname{"<}+ which is okay, BUT the last bracket is highlighted (highlighting of a single bracket without highlighting the whole pair, shouldn't even be possible | btw: there is a comment somewhere below the call stack of keypressevent, that clearMatches have to be called in qeditor at this position, perhaps this is also needed there).
\item  env highlighting doesn't work if there are points in the env. name (e.g. \verb+\begin{a..s}+)
\item completion and enter \verb+\\dr...+ (if you write \verb+\\+ and then something else an empty completion list is opened)
\item auto replace is not atomic for menu replace (e.g. select text, press shift+alt+f, multiple undos necessary to return to selection)
\item Home/Ende keys don't work (siehe mrunix forum)
\item adding new bib tex entries doesn't always update completer list;; also changing citations in latex doc also require bibtex call and aren't detectedd
\item completer doesn't contain citation labels (this means if you have \verb+\cite{abcd|+ with | marking cursor position it doesn't suggest abcdef as citation even if \verb+\cite{abcd}+ is in the list)
\item referenz/label/citation doesn't check for verb e.g. \verb+\verb#\cite{abc}\ref{xx}\label{xx}#+
\item completer list for references shows label defined in comments (e.g. \verb+%\label{sssss}+) if there is a master document set
\item setting a master document clears the current completer reference list
\item hint by  \verb+\ref+  can be much to long (e.g. write \verb+\re+, the hint for the first reference is shown and can hide the completer list, if there is a lot of text in the line after the corresponding label)
\item preview and beamer class (preview image contains slide template, but ">verzerrt"<)
\item preview doesn't always work (e.g. no tooltip appears, not 

	\item Wrong indentation of \begin{verbatim}
		@MastersThesis{Marold,
		author = {Alexander Marold},
		title = {Rekonstruktion 
		der globalen 
		Ereignisreihenfolge aus lokalen Logdateien},
		school = {Heinrich-Heine-Universität Düsseldorf},
		year = {2008},
		type = {Bachelorarbeit},
		}
	\end{verbatim} in bib files
\end{itemize} 

\subsubsection{others}
\begin{itemize}
	\item QCodeEdit: \begin{itemize}
		\item Deleting placeholders by pressing repeadetly backspace after one
		\item if replace all is restarted from scope begin, show the count of replaced occurrences before the scope restart and afterwards.
		\item show tooltips when scrolling (line, section, chapter)
		\item RTL support like in texmaker 1.9.0/1 (but there it was removed in 1.9.2) (import fy to unicode)
		\item BUG? (probably fixed already): 
		search something below the current visible area, scroll up, search again, press arrow down, the search result is drawn selected although it isn't
	\end{itemize}
	\item Latex Parser: \begin{itemize}
		\item move everything in the \verb+Latexparser+ class
		\item NextWord with cookie state (to recognize multi line verbatim)?? Or read from qce?
		\item Understand some TeX Things (e.g. \verb+^^5c^^5c+ equals \verb+\\+ if not in verbatim)
	\end{itemize}
	\item better structure view: \begin{itemize}
		\item select section title when double clicked of one?		
		\item parsing not loaded files
		\item show section/figure/table numbers
		\item should cache old parsing, don't reparse unmodified files
		\item could use .aux files
		\item should not execute several regex after each other on the same line (custom parser? mixed regex?)	
	\end{itemize}
	\item Synonym dictionary: \begin{itemize}
		\item ignore space on words looked up
		\item remember window size
		\item allow selecting of words with arrow keys
		\item faster contains checks using Tries
		\item support for 
		\item adding own words like in the spell checking 
		\item online lookup (e.g german: wortschatz.uni-leipzig.de)
	\end{itemize}
	\item Labels/References: \begin{itemize}
		\item understand \verb+\nameref, \vref+
		\item allow to define custom commands
		\item 	show all refs using a label and allow easy jumping to them (e.g selection window?)
		\item using a SoundEx suggestion list for wrong references/labels like in the spell checker
		\item a reference/label check dialog like the old spell checking dialog
	\end{itemize}
	\item Citations: \begin{itemize}
		\item unit tests for bibtex parser
		\item show SoundEx suggestion for wrong citatinos
		\item check environment paths (e.g. \verb+BIBINPUTS = C:\My Documents\CDMA\Papers\Bib;C:\My Documents\GPS\Papers\Bib+)
	\end{itemize}
	\item Better new document wizard (look at kile)
	\item Completion List \begin{itemize}
		\item treat \verb+\+abc\{\%<xyz\%>\} and \verb+\+abc\{\%<def\%>\} as identical and ignore one of thems
		\item Understand \verb+\usepackage}, newcounter, tex: \newcount \newfont \def, \edef, \xdef, \gdef+
		\item text completion should show parameters of commands (e.g. reference names) even if they aren't currently used
		\item merging text completion and spell checker cache?
	\end{itemize}
	\item GUI: \begin{itemize}
		\item different categories for the formats in the formatconfig
		\item search for latex commands in the completion package list; list for every available package the contained commands
		\item using new tm 1.9.2 icons
		\item new tm 1.9 output view (icons left instead of tabs + next/previous error buttons)
		\item using icons for math font styles (like in text font style)
		\item Customizable symbol lists
		\item combo boxes (dropdown tool buttons) instead of several icons for each command (like 1.9 or kile)
	\end{itemize}
	\item Preview: \begin{itemize}
	
		\item allow real zoom (regenerating png)
	\end{itemize}
	\item  Checker: \begin{itemize}
		\item Silbentrennung: \verb+http://homepage.ruhr-uni-bochum.de/Georg.Verweyen/silbentrennung.html+
		\item Respect \verb+\verb+ or even multiline \verb+\begin{verbatim}+
		\item Grammar Checker
		\item LaTeX Checker	(calling lacheck?)
	\end{itemize}
	\item Log files:\begin{itemize}
	\item if there are several errors, should it jump to the first or the one after the current cursor position? If the errors are in different files, should it jump to another file? which one? Should it jump to warnings? 	(current in r500 it only jumps to the next error in the same file) 
		\item More log: What do we do with errors in other files when we are not in master mode? 
		\item Is it a good idea to change the tab from ">log file"< to ">messages"< if another tex file is opened and we are not in master mode?
		\item don't jump away from log file (what did that mean???????)
	\end{itemize}
	\item Building \begin{itemize}
		\item Don't require to save everything (temporary location?)
		\item Asymptote support
		\item Is the output of (pdf)latex shown? should it?
		\item Improved recognizing of tool paths, checking of correct tool settings,  custom build actions: planned (although former is finished for miktex+ghostscript)
		\item better browser detection in WebPublish dialog 
		\item running arbitrary commands as CMD\_UNKNOWN
		\item better user command (allow to run the other programs as intern://latex, intern://view/dvi, intern://showlog ?)
		\item if the parameters of the user and the program differ, which should be used? (that problem is the reason, ghostscript have to be the plain program name)
		\item . command tooltips (right section?)
	\end{itemize}
	\item Configuration \begin{itemize}
		\item open combobox for shortcut changing if double clicked on standard shortcut
		\item use some macro tricks, so that not everything has to be written five times (header declaration, loading, storing, config dialog in, config dialog out)
		\item update all overlays (environment, spell checking, refs..) when they are en/disabled in the config dialog (? will be slow? disabling => just remove overlays instead reparsing)
	\end{itemize} 
	\item  Scripting
	\begin{itemize}
	\item qdocumentsearch access
	\item universalinpuit dialog
	\end{itemize}
	\item Text analyse should have a list of environments to ignore and if it should read math/envs
	\item Plugins
	\item Recognizing hand written symbols (online lookup in detextify)
	\item New Tabular Dialog (of kile)
	\item Having functions to select environments/blocks/sections, changing \verb+$$+ to \verb+\[\]+, or to \verb+\begin{something}+ (waiting for qce2.3 syntax engine)
	\item single instance: open the documents in the order the second instances were called instead in the order theirs messages arrive in the first instance (using a queue like described in 2877037)
	\item Unit Tests (gui, qeditor, qdocument,..)
\end{itemize}


\end{document}

